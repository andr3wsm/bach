\chapter{Progettazione logica}

Lo schema concettuale presentato nel Capitolo \ref{conceptual} rappresenta in modo accurato il dominio a livello astratto ma la traduzione in uno schema logico nel modello relazionale richiede la semplificazione di alcune strutture concettuali.
In seguito a questo passaggio di ristrutturazione si definisce uno schema logico, costituito da un insieme di relazioni con eventuali vincoli posti su di esse.

\section{Ristrutturazione dello schema Entità-Relazione}
\label{restructuring}

Il formalismo del modello Entità-Relazione permette di utilizzare strutture piuttosto complesse a livello concettuale, come ad esempio le relazioni di specializzazione o gli attributi multivalore.
Per facilitare la traduzione in uno schema logico nel modello relazionale è necessario operare una ristrutturazione o semplificazione dello schema iniziale \cite{Atz23}, con l'obiettivo di ottenere soltanto entità senza specializzazioni o categorie, attributi semplici e relazioni binarie.
Nella fase di ristrutturazione si valuta se è opportuno mantenere gli attributi ridondanti per motivi di efficienza e si scelgono gli attributi chiave delle entità che hanno più candidati possibili.

Lo schema ristrutturato, mostrato in Figura \ref{completererdiagram}, rimane espresso nel modello Entità-Relazione, ma non si può più considerare puramente concettuale perché introduce modifiche dettate non dalle caratteristiche del dominio ma da aspetti implementativi e ottimizzazione delle prestazioni.
Dopo la ristrutturazione, la traduzione può essere fatta in modo meccanico perché i costrutti più elementari del modello Entità-Relazione hanno dei corrispettivi diretti nel modello relazionale.

\subsection{Ristrutturazione dell'area concettuale della gravidanza}

La Figura \ref{pregnancyrerdiagram} mostra l'ingrandimento dello schema Entità-Relazione ristrutturato per quanto riguarda le entità Paziente, Gravidanza e Malattia.
Di queste solo Gravidanza subisce delle modifiche rispetto allo schema concettuale.

\subsubsection{Gravidanza}

Nell'entità Gravidanza semplifichiamo gli attributi composti in attributi semplici, legandoli quindi direttamente alla gravidanza.
Viene mantenuto il vincolo posto sulla non decrescenza o crescenza stretta dei componenti dell'attributo Parità.

Gravidanza è entità debole ed è identificata dalla relazione con Paziente e nello schema concettuale sono state identificate più possibili chiavi parziali, ovvero la parità e la data di primo ingresso.
Considerando che la maggior parte delle entità dell'intero schema sono deboli, direttamente o indirettamente, rispetto alla gravidanza, si otterrà che nello schema logico (vedi Sezione \ref{logical}) in quasi tutte le tabelle sarà presente una chiave esterna che fa riferimento alla gravidanza.
Per questo motivo, al fine di alleggerire il numero di colonne presenti in ciascuna tabella e facilitare l'interrogazione, introduciamo un valore numerico identificativo come attributo chiave dell'entità Gravidanza ristrutturata.

La parità e la data del primo ingresso, in quanto chiavi candidate nel modello concettuale, mantengono le proprietà di chiave e quindi viene posto un ulteriore vincolo di unicità\footnote{
    Il vincolo di non avere valore \emph{NULL} è già espresso dall'assenza dell'indicazione (0,1) su tutti gli attributi, quindi nessuno di essi è opzionale.
}.
\begin{itemize}
\item[\Con{}] La quadrupla di attributi che compongono la parità deve essere unica per ogni paziente.
\item[\Con{}] La data di primo ingresso deve essere unica per ogni paziente.
\end{itemize}

L'attributo composto relativo alla PMA viene scomposto in due attributi, Tipo e Ovodonazione, su cui si pone un ulteriore vincolo.
\begin{itemize}
\item[\Con{}] L'attributo Tipo PMA è presente se e solo se è presente l'attributo Ovodonazione PMA.
\end{itemize}

\subsection{Ristrutturazione dell'area concettuale delle visite}

La Figura \ref{visitsrerdiagram} mostra l'ingrandimento dello schema Entità-Relazione ristrutturato per quanto riguarda le entità relative alle visite e all'entità Esame.
Quest'ultima non subisce modifiche nella ristrutturazione.

\subsubsection{Visite}

Considerando la presenza di molti attributi in comune, le diverse entità coinvolte nella categoria Visita collassano in un'unica entità.
Si introduce quindi un attributo ulteriore che indica a quale delle quattro classi di visite appartiene ogni istanza dell'entità Visita.

L'entità Visita rimane debole rispetto a Gravidanza, come erano tutte le diverse visite presenti nello schema concettuale, introducendo come chiave parziale la data per tutte.
L'attributo \enquote{Anomalie morfologiche fetali} della visita del primo trimestre e l'equivalente \enquote{Anomalie fetali} della visita del secondo trimestre confluiscono in un unico attributo.
L'attributo BMI, calcolabile facilmente a partire da peso e altezza, è ridondante e si può rimuovere.

Come conseguenza della semplificazione della categoria, la relazione che Visita ha con Esame modifica la partecipazione rendendola parziale, dato che non è richiesto che ad ogni visita vengano effettuati esami.
Le diverse relazioni che legavano le entità originali a Gravidanza si riducono a una sola relazione sulla quale si pongono nuovi vincoli per rispettare la cardinalità imposta dalle relazioni originali.
\begin{itemize}
\item[\Con{}] Ogni gravidanza può effettuare al massimo una sola visita del primo trimestre.
\item[\Con{}] Ogni gravidanza può effettuare al massimo una sola visita del secondo trimestre.
\end{itemize}

\subsection{Ristrutturazione dell'area concettuale del parto}

La Figura \ref{deliveryrerdiagram} mostra l'ingrandimento dello schema Entità-Relazione ristrutturato per quanto riguarda le entità parto, con le sue specializzazioni, e Travaglio.
Entrambe subiscono modifiche rispetto allo schema originale.

\subsubsection{Parto}

Nello schema concettuale originale l'entità Parto è specializzata in più entità su due livelli.
Consideriamo il primo livello di specializzazione, articolata in parti cesarei programmati e parti di altro tipo, che è totale e disgiunta: ristrutturiamo la specializzazione con un'entità Parto, che contiene gli attributi comuni, e le due entità Cesareo programmato e Parto con travaglio, deboli rispetto a Parto, che contengono gli attributi specifici.
Per mantenere le proprietà previste dalla specializzazione, ovvero totalità e disgiunzione, è necessario introdurre un vincolo ulteriore.
\begin{itemize}
\item[\Con{}]  Ogni parto deve essere uno e uno solo tra cesareo programmato e parto con travaglio.
\end{itemize}

Il secondo livello di specializzazione, che distingue parti naturali, operativi e cesarei urgenti, è totale e sovrapposta, con solo pochi attributi che non sono in comune a tutte le sottoentità.
Eliminiamo quindi la specializzazione, legando anche gli attributi specifici delle sottoentità a Parto con travaglio.
Si impongono i seguenti vincoli.
\begin{itemize}
\item[\Con{}] L'attributo Motivo è presente se e solo se il parto è operativo o cesareo.
\item[\Con{}] L'attributo Kristeller è presente se e solo se il parto è naturale o operativo.
\end{itemize}

Gli attributi composti sia nell'entità Parto (Secondamento) sia nell'entità Parto con travaglio (Tempi) vengono scomposti nei relativi attributi che li compongono.

\subsubsection{Travaglio e induzione}

L'attributo composto e multivalore Induzione diventa un'entità a sé stante, con l'istante di somministrazione come chiave parziale.
L'entità Travaglio rimane senza nessun attributo, quindi la rimuoviamo per legare direttamente Parto con travaglio e Induzione; l'entità Induzione diventa quindi debole rispetto a Parto con travaglio.

\subsection{Ristrutturazione dell'area concettuale del neonato}

La Figura \ref{newbornrerdiagram} mostra l'ingrandimento dello schema Entità-Relazione ristrutturato per quanto riguarda le entità Neonato, Tracciato CTG e Misurazione CTG.
Le differenze rispetto allo schema originale sono minime: nell'entità Neonato l'attributo composto APGAR viene diviso nelle sue componenti che vengono collegate direttamente a Neonato, mentre nell'entità Tracciato CTG si sceglie come chiave la relazione identificante con l'entità Parto.

\begin{sidewaysfigure}
    \centering
    \includesvg{vectors/modello-er-ristrutturato.svg}
    \caption{Schema Entità-Relazione ristrutturato. Visione d'insieme semplificata, contenente soltanto entità e relazioni del modello. Le aree tratteggiate delimitano gli stessi ingrandimenti dello schema concettuale originale.}
    \label{completererdiagram}
\end{sidewaysfigure}

\begin{sidewaysfigure}
    \centering
    \includesvg{vectors/modello-er-ristrutturato-gravidanza.svg}
    \caption{Schema Entità-Relazione ristrutturato. Ingrandimento dell'area gravidanza.}
\label{pregnancyrerdiagram}
\end{sidewaysfigure}

\begin{sidewaysfigure}
    \centering
    \includesvg{vectors/modello-er-ristrutturato-visite.svg}
    \caption{Schema Entità-Relazione ristrutturato. Ingrandimento dell'area visite.}
\label{visitsrerdiagram}
\end{sidewaysfigure}

\begin{sidewaysfigure}
    \centering
    \includesvg{vectors/modello-er-ristrutturato-parto.svg}
    \caption{Schema Entità-Relazione ristrutturato. Ingrandimento dell'area parto.}
\label{deliveryrerdiagram}
\end{sidewaysfigure}

\begin{sidewaysfigure}
    \centering
    \includesvg{vectors/modello-er-ristrutturato-neonato.svg}
    \caption{Schema Entità-Relazione ristrutturato. Ingrandimento dell'area neonato.}
\label{newbornrerdiagram}
\end{sidewaysfigure}

\FloatBarrier

\section{Schema logico relazionale}
\label{logical}

Lo schema logico è un insieme di relazioni composte di attributi, corrispondenti concettualmente a tabelle strutturate in colonne; ogni relazione può essere istanziata come insieme di tuple, corrispondenti all'insieme dei record nelle tabelle.
Tra diverse tabelle possono sussistere vincoli di chiave esterna, ovvero situazioni in cui ogni istanza di un dato attributo di una relazione deve essere presente come istanza di attributo dell'altra relazione coinvolta.

La traduzione da schema concettuale ristrutturato (Sezione \ref{restructuring}) a schema logico è molto lineare perché il primo dei due viene elaborato appositamente, a partire dallo schema concettuale originale (Sezione \ref{conceptual}), per avvicinarsi alla modellazione tipica degli schemi relazionali.
In generale tutte le entità vengono mantenute con i rispettivi attributi, andando a costituire \enquote{relazioni}; tra le relazioni concettuali, invece, solo le relazioni molti a molti costituiscono relazioni logiche vere e proprie, mentre le altre fanno confluire i loro attributi nelle tabelle delle entità dal lato in cui hanno molteplicità uno.

La Figura \ref{completereldiagram} mostra uno schema relazionale di massima contenente soltanto i nomi delle relazioni e i vincoli di chiave esterna tra relazioni.
Gli schemi relazionali usano la seguente sintassi: gli attributi chiave sono indicati con l'icona della chiave; gli attributi opzionali sono indicati in corsivo; i vincoli di chiave esterna sono rappresentati con la notazione \emph{crow's foot}\footnote{
    La notazione \emph{crow's foot} indica la cardinalità delle relazioni tra entità (nel modello Entità-Relazione) o tra tabelle (nel modello relazionale). Una linea singola indica cardinalità 1, mentre il simbolo della \enquote{zampa di corvo}, che rappresenta idealmente un collegamento con più punti terminali, indica cardinalità N.
}.

\subsection{Area della gravidanza}
\label{logicalpregnancy}

Le entità Paziente, Malattia e Gravidanza costituiscono tabelle a sé stanti.
La relazione \enquote{ha} viene conservata nell'attributo \at{paziente\_cf} della tabella \tab{gravidanza}, mentre la relazione \enquote{avuta in concomitanza con}, che lega gravidanze e malattie, viene memorizzata in una tabella separata.
Sintetizziamo di seguito i vincoli precedentementi posti sullo schema concettuale (Sezione \ref{conceptual} e Sezione \ref{restructuring}) e riformulati come vincoli sullo schema logico.
\begin{itemize}
\item[\Con{}] La coppia (\at{paziente\_cf}, \at{data\_primo\_ingresso}) di \tab{gravidanza} è unica.
\item[\Con{}] La tupla (\at{paziente\_cf}, \at{figli\_nati\_vivi}, \at{figli\_nati\_pretermine}, \at{figli\_nati\_a\_termine}, \at{aborti\_avuti}) di \tab{gravidanza} è unica.
\item[\Con{}] Se una paziente ha più di una gravidanza, per ogni coppia di gravidanze successive i valori di \at{figli\_nati\_vivi}, \at{aborti\_avuti}, \at{figli\_nati\_pretermine}, \at{figli\_nati\_a\_termine} sono non decrescenti e almeno uno tra essi è crescente.
\item[\Con{}] In \tab{gravidanza} l'attributo \at{pma\_tipo} è presente se e solo se l'attributo \at{pma\_ovodonazione} è presente.
\end{itemize}

\subsection{Area delle visite}
\label{logicalvisits}

Le entità Visita e Esame costituiscono tabelle a sé stanti. La relazione molti a molti \enquote{fatto per} diventa una tabella separata che lega esami e visite.

% da chiarire
Introduciamo un attributo \at{tipo} alla relazione \tab{esame} così ottenuta che descrive il tipo di dato dei valori di \at{esito} in \tab{esame\_visita}.
Questo è utile a livello implementativo perché il tipo di dato dell'esito di un dato esame può variare in base all'esame stesso.
L'attributo \at{categoria\_visita} di \tab{visita} ha uno dei seguenti valori: \val{primo\_trimestre}, \val{secondo\_trimestre}, \val{biometrica}, \val{altro\_tipo}.
L'attributo \at{stato\_crescita} di \tab{visita}, se non è \val{NULL}, ha uno dei seguenti valori: \val{regolare}, \val{fgr}, \val{sga}.

Riassumiamo di seguito i vincoli da porre su tale schema logico.
\begin{itemize}
\item[\Con{}] Per ogni gravidanza, in \tab{visita} esiste al massimo un solo record che ha valore \val{primo\_trimestre} nel campo \at{categoria\_visita}.
\item[\Con{}] Per ogni gravidanza, in \tab{visita} esiste al massimo un solo record che ha valore \val{secondo\_trimestre} nel campo \at{categoria\_visita}.
\item[\Con{}] Per ogni gravidanza deve essere presente almeno un record in \tab{visita} oppure in \tab{parto}.
\item[\Con{}] Per ogni esame registrato per una data visita, l'esito dell'esame deve essere compatibile con il \at{tipo\_esito} indicato nel record corrispondente nella tabella \tab{esame}.
\end{itemize}

\subsection{Area del parto}
\label{logicaldelivery}

Le entità Parto, Cesareo programmato, Parto con travaglio e Induzione formano ciascuna una tabella a sé stante mentre tutte le relazioni vengono rimosse e si introduce come attributo chiave (o parte della chiave) l'identificatore della gravidanza in tutte le tabelle.

L'attributo \at{tipo\_parto} di \tab{parto} deve essere uno tra \val{cesareo\_programmato} e \val{parto\_con\_travaglio}.
Di conseguenza si impone il seguente vincolo.
\begin{itemize}
\item[\Con{}] Il valore di \at{tipo\_parto} è \val{cesareo\_programmato} se e solo se il parto ha un record corrispondente in \tab{cesareo\_programmato}.
\item[\Con{}] Il valore di \at{tipo\_parto} è \val{parto\_con\_travaglio} se e solo se il parto ha un record corrispondente in \tab{parto\_con\_travaglio}\footnote{
    Questo vincolo è equivalente e soddisfa il seguente: \enquote{Ogni Parto è uno e uno solo tra cesareo programmato e parto con travaglio}.
}.
\end{itemize}

L'attributo \at{sottotipo\_parto} di \tab{parto\_con\_travaglio} deve essere uno tra \val{naturale}, \val{operativo}, \val{cesareo} (dove \enquote{urgente} è sottinteso per semplicità di rappresentazione), \val{naturale\_operativo},\linebreak \val{naturale\_cesareo}, \val{operativo\_cesareo}.
Si impongono quindi i seguenti vincoli.
\begin{itemize}
\item[\Con{}] L'attributo \at{motivo} di \tab{parto\_con\_travaglio} è assente se e solo se \at{sottotipo\_parto} ha valore \val{naturale}.
\item[\Con{}] L'attributo \at{kristeller} di \tab{parto\_con\_travaglio} è assente se e solo se \at{sottotipo\_parto} ha valore \val{cesareo}.
\end{itemize}

\subsection{Area del neonato}
\label{logicalnewborn}

Le entità Neonato, Tracciato e Misurazione costituiscono tabelle a sé stanti, mentre solo la relazione \enquote{fatta su} diventa una tabella a parte (con il nome di \tab{misurazione\_neonato}) e le altre vengono invece accorpate alle altre tabelle.

Il ciclo di relazioni che si forma a livello concettuale, comprendente le entità Parto, Neonato, Misurazione e Tracciato, non crea problemi di integrità perche la relazione di \enquote{contatto}, ovvero \enquote{fatta su}, coinvolge entità che sono tutte deboli rispetto a Parto; ciò permette di includere questa proprietà imponendo il vincolo di chiave della tabella \tab{misurazione\_neonato}.

È necessario imporre un ultimo vincolo.
\begin{itemize}
\item[\Con{}] Per ogni misurazione deve essere presente almeno uno tra l'attributo \val{valore\_fcm}, l'attributo \val{valore\_toco} e un record nella tabella \tab{misurazione\_neonato}.
\end{itemize}

\begin{figure}
    \centering
    \includesvg{vectors/modello-relazionale.svg}
    \caption{Schema relazionale. Visione d'insieme semplificata, contenente soltanto i nomi delle relazioni o tabelle dello schema e le relazioni di chiave esterna.}
\label{completereldiagram}
\end{figure}

\begin{sidewaysfigure}
    \centering
    \includesvg{vectors/modello-relazionale-gravidanza.svg}
    \caption{Schema relazionale. Ingrandimento dell'area gravidanza.}
\label{pregnancyreldiagram}
\end{sidewaysfigure}

\begin{sidewaysfigure}
    \centering
    \includesvg{vectors/modello-relazionale-visite.svg}
    \caption{Schema relazionale. Ingrandimento dell'area visite.}
\label{visitsreldiagram}
\end{sidewaysfigure}

\begin{sidewaysfigure}
    \centering
    \includesvg{vectors/modello-relazionale-parto.svg}
    \caption{Schema relazionale. Ingrandimento dell'area parto.}
\label{deliveryreldiagram}
\end{sidewaysfigure}

\begin{sidewaysfigure}
    \centering
    \includesvg{vectors/modello-relazionale-parto.svg}
    \caption{Schema relazionale. Ingrandimento dell'area neonato.}
\label{newbornreldiagram}
\end{sidewaysfigure}