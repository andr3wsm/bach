\chapter{Progettazione logica}

Lo schema concettuale presentato nel Capitolo \ref{conceptual} rappresenta in modo accurato il dominio a livello astratto ma la traduzione in uno schema logico nel modello relazionale richiede la semplificazione di alcune strutture concettuali.
In seguito a questo passaggio di ristrutturazione si definisce uno schema logico, costituito da un insieme di relazioni con eventuali vincoli posti su di esse.

\section{Ristrutturazione dello schema Entità-Relazione}

Il formalismo del modello Entità-Relazione permette di utilizzare strutture piuttosto complesse a livello concettuale, come ad esempio le relazioni di specializzazione o gli attributi multivalore.
Per facilitare la traduzione in uno schema logico nel modello relazionale è necessario operare una ristrutturazione o semplificazione dello schema iniziale \cite{Atz23}, con l'obiettivo di ottenere soltanto entità senza specializzazioni o categorie, attributi semplici e relazioni binarie.
Nella fase di ristrutturazione si valuta se è opportuno mantenere gli attributi ridondanti per motivi di efficienza e si scelgono gli attributi chiave delle entità che hanno più candidati possibili.

%% MANCA schema ristrutturato completo

Lo schema ristrutturato, mostrato in Figura \ref{}, rimane espresso nel modello Entità-Relazione, ma non si può più considerare puramente concettuale perché introduce modifiche dettate non dalle caratteristiche del dominio ma da aspetti implementativi e ottimizzazione delle prestazioni.
Dopo la ristrutturazione, la traduzione può essere fatta in modo meccanico perché i costrutti più elementari del modello Entità-Relazione hanno dei corrispettivi diretti nel modello relazionale.

\subsection{Ristrutturazione dell'area concettuale della gravidanza}


\begin{figure}
    \centering
    \includesvg{vectors/modello-er-ristrutturato-gravidanza.svg}
    \caption{Schema Entità-Relazione ristrutturato. Ingrandimento dell'area gravidanza.}
\label{pregnancyrerdiagram}
\end{figure}

La Figura \ref{pregnancyrerdiagram} mostra l'ingrandimento dello schema Entità-Relazione ristrutturato per quanto riguarda le entità Paziente, Gravidanza e Malattia.

\subsubsection{Gravidanza}

Nell'entità Gravidanza semplifichiamo gli attributi composti in attributi semplici, legandoli quindi direttamente alla gravidanza.
Si rimuove la \enquote{rappresentazione testuale}, presente nello schema concettuale come attributo derivato e direttamente calcolabile a partire dagli altri componenti dell'attributo composto Parità.
Viene mantenuto il vincolo posto sulla non decrescenza o crescenza stretta dei componenti dell'attributo Parità.

Gravidanza è entità debole ed è identificata dalla relazione con Paziente.
Tra le due chiavi parziali possibili si sceglie l'attributo Data primo ingresso perché è costituito da un solo attributo invece che da quattro diversi, risultando quindi molto più semplice da utilizzare come identificatore della gravidanza nelle diverse relazioni in cui questa entità partecipa.
La quadrupla di attributi risultante dalla decomposizione di Parità, in quanto chiave candidata nel modello concettuale, mantiene le proprietà di chiave e quindi si pone un ulteriore vincolo di unicità\footnote{
    Il vincolo di non avere valore \emph{NULL} è già espresso dall'assenza dell'indicazione (0,1) su tutti gli attributi, quindi nessuno di essi è opzionale.
}.
\begin{quote}
La quadrupla di attributi che compongono la parità deve essere unica per ogni paziente.
\end{quote}

\section{Schema logico relazionale}

