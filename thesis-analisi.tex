%% Capitolo
\chapter{Analisi del dominio e dei requisiti}

Il fulcro della progettazione di una base di dati è l'analisi dei requisiti, in quanto essi stabiliscono quali informazioni dovranno essere memorizzate e quali vincoli vengono posti su di esse \cite{Ded16}.
Questa fase deve essere condotta in modo accurato ed esaustivo perché eventuali errori commessi al momento della definizione dei requisiti richiedono una quantità molto più elevata di tempo e risorse per essere corretti rispetto a quelli introdotti successivamente, ad esempio nell'implementazione.
L'acquisizione dei requisiti deve superare il divario di conoscenza del dominio che si ha tra l'utente finale, in questo caso un membro del reparto di Ginecologia e Ostetricia che si interfaccerà con il sistema, e il progettista che lo deve realizzare; ciò si attua con interviste dirette ai committenti e con uno studio accurato dei sistemi già in uso, rilevando le esperienze e le eventuali difficoltà avute.

\section{Stato di fatto}
\label{problem}

Il personale medico utilizza sistemi informatici specifici, integrati e uniformati a livello di azienda sanitaria o di sistema sanitario regionale, per compilare e conservare referti di visite ed esami vari; ciò permette il funzionamento dei fascicoli sanitari digitalizzati attualmente in uso.
I software in questione sono progettati per rispettare normative legali e standard di qualità e privacy, consentendo di accedere ai singoli referti, ma rendono più difficoltosa l'estrazione delle singole informazioni su pazienti o esami.
Per i medici della Clinica che ha richiesto la realizzazione di un nuovo sistema informatico, la necessità è di poter inserire o visualizzare i dati in modo immediato, senza che sia necessario leggere l'intero referto, consentendo inoltre di processare i risultati a fini statistici anche con metodi automatizzati.

I diversi momenti nel corso della gravidanza in cui vengono registrati dati relativi alla paziente (come parametri fisiologici o esami svolti) si sintetizzano nei seguenti.

\begin{enumerate}
\item Una visita nel primo trimestre di gravidanza, svolta tra le 11° e la 13° settimana, volta principalmente all'esame chiamato \emph{translucenza nucale} e a un accertamento delle condizioni iniziali della gravidanza.
\item Una visita nel secondo trimestre, detta \emph{morfologica}, che punta ad analizzare lo sviluppo del feto, svolta tra la 19° e la 21° settimana.
\item Una o più visite ecografiche, dette anche \emph{biometriche}, svolte nel corso del terzo trimestre.
\item Il travaglio e il parto, con informazioni sull'intero processo inclusi dati sui neonati.
\end{enumerate}

È importante considerare che il personale medico registra dati soltanto per le visite svolte all'interno della Clinica, e accade frequentemente che una o più visite vengano effettuate in altre strutture o che anche il parto possa avvenire presso altri punti nascite.

Oltre agli esiti di queste visite si vorrebbero conservare informazioni su eventuali patologie delle pazienti, potendo contare sulla presenza di un ambulatorio specifico per le gravidanze \emph{a rischio} che monitora anche le terapie seguite.

\section{Tecnologie e collezioni di dati in uso}
\label{usednow}

Come presentato nella Sezione \ref{problem} i medici lamentano la mancanza di un software, personalizzato e parallelo ai sistemi di refertazione \enquote{ufficiali} attualmente in uso, che favorisca la lettura dei dati ai fini di consultazione e analisi statistica sull'insieme di pazienti.
In assenza di uno strumento centralizzato adatto, nel corso degli ultimi anni (principalmente a partire dal 2022-2023) il personale della Clinica ha registrato i dati di interesse attraverso altri strumenti, affiancati quindi ai referti, variabili sia nel livello di strutturazione dei dati sia nelle modalità di interazione.

Le informazioni raccolte dalle visite vengono registrate principalmente in semplici fogli di calcolo tabellari, accessibili dai diversi calcolatori del reparto e compilati manualmente: elencheremo i diversi campi di queste tabelle con il relativo significato e i possibili valori che possono assumere.
A questi si aggiunge un software più complesso e strutturato, ma giudicato non adatto alle esigenze del personale medico.

\subsection{Tabella per la visita del primo trimestre}
\label{firsttrimester}

Nella visita del primo trimestre vengono annotati aspetti di base ma fondamentali della gravidanza, insieme a primi test genetici e di rischio per determinate condizioni patologiche.
In questa visita viene svolto l'esame della translucenza nucale, volto a rilevare malformazioni nel feto già negli stadi iniziali della gravidanza \cite{Sou05}.
Questa tabella, così come le successive, contiene un record per ogni gravidanza.

Nella tabella la notazione usata adotta frequentemente abbreviazioni e i campi indicati sono stati predisposti direttamente dal personale medico che poi li deve compilare, adattandosi quindi alle convenzioni a cui gli operatori sono abituati.
La compilazione manuale dà ovviamente la possibilità di inserire informazioni sbagliate in tabella, siano esse dati verosimili ma non corrispondenti alla realtà oppure dati corretti ma espressi in modo inconsistente.

\begin{itemize}
\item \column{Data NT}: data della visita. La sigla NT sta per \enquote{translucenza nucale} (\emph{nuchal translucency}).
\item \column{Cognome}, \column{Nome}, \column{DdN}: dati personali identificativi della paziente (cognome, nome e data di nascita).
\item \column{Età all'esame}: età della paziente, espressa come numero intero.
\item \column{Gemellare}: indica se la gravidanza è gemellare, espresso con valori 0/1.
In questo attributo e nei seguenti, dove non espresso diversamente, 0 indica un valore falso, assente o negativo e 1 indica un valore vero, presente o positivo.
\item \column{EG}: età gestazionale, indicata come settimane più giorni (ad esempio 12+5).
\item \column{Rischio T21, Rischio T18, Rischio T13}: rischio valutato per le trisomie 21, 18 e 13.
I fattori di rischio sono indicati usualmente come classi di rischio, che vanno da alto (A), medio (M oppure I, intermedio), basso (B).
In alcuni casi il rischio viene indicato in modo più preciso come rapporto 1:X, con X valore intero, e un livello basso corrisponde a un rapporto 1:1000 o inferiore, un livello medio si trova tra 1:250 e 1:1000, un livello alto è 1:250 o superiore.
La scelta di indicare il rischio con una lettera, con l'intera parola o come rapporto dipende dall'operatore che compila la tabella.
\item \column{Morfologia fetale}: indica se sono presenti anomalie nella morfologia fetale, con valori 0/1.
\item \column{Anomalie segnalate}: se il campo precedente ha valore 1, indica le anomalie morfologiche riscontrate come descrizione testuale.
\item \column{NT}: esito dell'esame della translucenza nucale, indica lo spessore della plica nucale in millimetri espresso come numero reale. Il risultato può essere accompagnato da una descrizione testuale.
\item \column{Esito genetica}: risultati di test genetici, eseguiti opzionalmente, successivamente alla visita. I risultati sono molto variabili e indicati come descrizione testuale.
\item \column{Premorfo indicata?}: inidca se viene consigliata un'ecografia premorfologica, con valori 0/1 ed eventualmente la motivazione nel caso 1.
\item \column{NIPT}: riporta se è stato effettuato il test prenatale non-invasivo (\emph{Noninvasive Prenatal Testing}).
\item \column{esito NIPT}: esito dell'esame NIPT, riportato come descrizione testuale.
\item \column{BS/BSOB}: indice morfologico misurato sulle dimensioni del tronco encefalico (BS) e della distanza dall'osso occipitale (BSOB), indicato come valore reale.
\item \column{CRL}: indice morfologico, misurato sulla lunghezza del feto in millimetri.
\item \column{PAPP-A}, \column{freeBHCG}, \column{PLGF}: esami svolti sul sangue della paziente per la rilevazione di proteine e materiale genetico, danno valori reali.
\item \column{UTPI}: indice di pulsatilità dell'arteria uterina, espresso come valore reale.
\item \column{PAM}: pressione arteriosa media materna, misurata come valore intero in mmHg.
\item \column{Esito <34}, \column{Esito <37}: misure della predisposizione alla preeclampsia, rispettivamente entro le 34 e le 37 settimane di gravidanza, espresse come rapporto di rischio.
\item \column{Prescrizione ASA}: indica se c'è stata prescrizione di cardioaspirina, con valori 0/1.
\item \column{Spontanea}, \column{PMA}, \column{IUI}, \column{FIVET}, \column{ICSI}, \column{Ovodonazione}: informazioni su concepimento ed eventuale tipo di procreazione medicalmente assistita, tutti con valori 0/1.
I campi \column{Spontanea} e \column{PMA} sono mutuamente esclusivi; gli altri campi, da \column{IUI} a \column{Ovodonazione}, sono compilati solo nel caso di procreazione medicalmente assistita, con \column{IUI}, \column{FIVET} e \column{ICSI} mutuamente esclusivi.
\item \column{Peso}, \column{Altezza}, \column{BMI}: parametri fisici della paziente alla visita.
\item \column{Fumo}: indica se la paziente fuma attualmente, con valori 0/1.
\item \column{Diabete pregestazionale}: indica se la paziente soffre di diabete pregestazionale (precedente o non dovuto alla gravidanza).
\item \column{Malattie autoimmuni}: indica, se ne soffre, le malattie autoimmuni della paziente, elencate in forma testuale.
\item \column{Pregressa PE}: indica se ha sofferto di preeclampsia (PE) in gravidanze precedenti.
\item \column{PE}, \column{IUGR}, \column{Eseguita ASA}, \column{EG al parto}: attributi predisposti per la compilazione alla conclusione del percorso di gravidanza, dato che si riferiscono a dati registrati alla fine della gravidanza o al momento del parto. In realtà questi campi normalmente non vengono compilati e le informazioni si possono reperire nella tabella che contiene i dati del parto.
\item \column{Note}: annotazioni aggiuntive.
\end{itemize}

\subsection{Tabella per la visita del secondo trimestre}
\label{secondtrimester}

La visita del secondo trimestre è incentrata sull'esecuzione di un'ecografia morfologica, ovvero volta a esaminare lo sviluppo del corpo del feto.
Si registrano anche eventuali esami svolti nel periodo precedente alla visita.

\begin{itemize}
\item \column{Data Morfo}: data di svolgimento della visita.
\item \column{Cognome}, \column{Nome}, \column{DdN}: dati personali identificativi della paziente.
\item \column{EG}, \column{Età all'esame}: età gestazionale ed età della paziente al momento della visita.
\item \column{Premorfo}: indica se è stata svolta un'ecografia premorfologia e l'evetntuale esito in forma testuale.
\item \column{Gemellare}: indica se la gravidanza è gemellare, con valori 0/1.
\item \column{Morfologia fetale}: indica se sono presenti anomalie morfologiche nel feto, con valori 0/1.
\item \column{Anomalie segnalate}: se il valore del campo precedente è 1, elenca le anomalie riscontrate in forma testuale.
\item \column{Decorso}: indica, in forma testuale, se ci sono stati eventi degni di nota nel periodo precedente alla visita.
\item \column{Esito genetica}: risultati di test genetici, non sempre presenti, descritti in forma testuale.
\item \column{Note}: annotazioni aggiuntive
\item \column{NIPT}: indica se è stato effettuato il test NIPT, con valori 0/1.
\item \column{Esito NIPT}: esito dell'esame NIPT. Come valori si trovano 0 (o equivalentemente: B, BR, basso, basso rischio), insufficiente (materiale mancante per dare un esito chiaro), oppure una descrizione testuale del risultato.
\end{itemize}

\subsection{Tabella per il parto}
\label{delivery}

Durante il ricovero ospedaliero per il periodo di travaglio e parto si registrano dati relativi sia al processo di espulsione (comprese tempistiche o complicazioni) sia parametri fisiologici della partoriente e dei nascituri.
I valori delle colonne da \column{APGAR} a \column{Sesso neonato} sono riferite ai neonati: se il parto è gemellare sono presenti due (o eventualmente più) valori, separati da uno spazio, per ciascuna di queste colonne, mantenendo ovviamente lo stesso ordine.

\begin{itemize}
\item \column{Data}: data del parto.
\item \column{PZ}: cognome e nome della paziente.
\item \column{S.G.}: età (stato) gestazionale, espresso sempre come settimane+giorni.
\item \column{Parità}: stringa di cifre che sintetizzano le gravidanze della paziente che hanno preceduto quella corrente.
Si considerano: il numero di figli nati a termine, il numero di figli nati pretermine, il numero di aborti (spontanei cossì come per interruzione volontaria di gravidanza), il numero totale di figli nati vivi; la stringa risultante è costituita di questi valori, nell'ordine riportato (ad esempio 1021, 0000), e quindi è costituita generalmente di 4 caratteri, ma occasionalmente può averne di più se la paziente ha avuto un numero notevole di gravidanze.
\item \column{Travaglio}: indica il metodo di induzione del travaglio. Può avere i seguenti valori: indotto, spontaneo, pilotato, senza travaglio. L'ultimo caso corrisponde a parti cesarei programmati.
\item \column{Motivo induzione}: indica il motivo dell'induzione del travaglio come descrizione testuale.
\item \column{Metodo induzione}: indica il metodo di induzione del travaglio. Solitamente corrisponde a una lista di abbreviazioni che rappresentano farmaci o metodi meccanici di induzione.
\item \column{Parto}: indica il tipo di parto, con i seguenti valori possibili: spontaneo, cesareo, operativo.
\item \column{Motivo parto operativo}: se il tipo di parto è operativo, riporta i motivi dell'attuazione, come descrizione testuale.
\item \column{Se TC}: se il tipo di parto è cesareo, riporta se è stato programmato come tale, con i seguenti valori possibili: programmato, urgente.
\item \column{Motivo TC}: se il tipo di parto è cesareo, riporta i motivi dell'attuazione, come descrizione testuale.
\item \column{Episiotomia}: indica se è stata svolta l'episiotomia, con valori possibili Sì o No.
\item \column{Motivo episiotomia}: se è sstata eseguita l'episiotomia, ne riporta il motivo come descrizione testuale.
\item \column{Lacerazioni}: indica il grado di lacerazioni. I valori possibili sono i seguenti: 0 (equivalenti: No, vuoto), 1, 2, 3, 4, AL (altro). Si applica a tutti i parti eccetto quelli cesarei programmati.
\item \column{Tracheloraffia}: indica la presenza di lacerazioni nel collo dell'utero, con valori possibili Sì o No.
\item \column{Perineo integro}: indica se il perineo è rimasto integro, con valori Sì o No.
\item \column{Secondamento}: indica la modalità di espulsione della placenta nel processo del parto. I valori possibili sono i seguenti: attivo, strumentale, manuale, scovolamento.
\item \column{Perdite}: indica la perdita ematica sofferta durante il parto, quantificata in mL.
\item \column{Robson}: classificazione del parto con valori della scala di Robson\footnote{
    Valori possibili nella classificazione di Robson: 1, 2a, 2b, 3, 4a, 4b, 5.1, 5.2, 6, 7, 8, 9, 10.
}.
\item \column{Analgesia}: indica se è stata somministrato un analgesico, con valori Sì o No.
\item \column{Tipo analgesia}: indica il tipo di analgesia somministrata. I valori possibili sono: spinale, epidurale (equivalente: peridurale), spinale e peridurale, calinox.
\item \column{APGAR}: valore di \enquote{vitalità} del neonato su una scala di valori interi da 0 a 10. Viene misurato al 1° minuto dalla nascita, al 5° minuto e occasionalmente anche al 10° minuto (ad esempio, 7/7/8).
\item \column{TIN}: indica se il neonato è stato sottoposto a terapia intensiva neonatale, con valori possibili Sì o No.
\item \column{pH}, \column{BE}: risultati delle analisi eseguite sul sangue del cordone ombelicale, espressi come valori reali.
\item \column{Sesso neonato}: indica il sesso attribuito al neonato, con valori possibili M o F.
\end{itemize}

\subsection{Rilevazione del cardiotocografo}
\label{cardiotocograph}

Durante il travaglio viene usato il cardiotocografo, uno strumento che rileva il battito cardiaco del feto (o anche di più feti nel caso di parti gemellari) e l'entità delle contrazioni uterine \cite{Ayr18}.
Questa apparecchiatura registra e salva i dati, in modo da poterli esportare in forma grafica o tabellare.
In particolare consideriamo il file risultato dall'esportazione in forma di foglio di calcolo, in cui vengono indicati alcuni dati identificativi della paziente.

\begin{itemize}
\item \column{Nome}, \column{Cognome}: informazioni personali della paziente.
\item \column{Num. progressivo}: identificativo progressivo, indicato come numero intero, attribuito dalla macchina alla registrazione.
\item \column{Data e ora inizio tracciato}: timestamp (in formato leggibile) del momento che la macchina assume come \enquote{0 secondi}.
\end{itemize}

I diversi sensori dello strumento sono sincronizzati e rilevano i rispettivi valori 4 volte al secondo.
La tabella contiene quindi una riga per ogni rilevazione, ciascuna delle quali ha i seguenti campi.

\begin{itemize}
\item \column{Secondi}, \column{Decimi}: indicazione del momento di rilevazione, rispetto al tempo di inizio tracciato indicato all'inizio della tabella. Entrambi i valori sono interi, con \column{Decimi} che indica i centesimi~[\textit{sic}] di secondo (0, 25, 50, 75).
\item \column{FCF}, \column{FCF (2)}, \column{FCF (3)}: indicano la frequenza cardiaca fetale, come valore intero, rilevata da ciascuno dei tre sensori di cui è dotato lo strumento. Anche se la gravidanza è singola, non è necessariamente il sensore 1 a rilevare il battito del feto. Un valore assente, sia per sensori non utilizzati sia per eventuali errori nella rilevazione, è indicato nella tabella con -1.
Nel caso di parti gemellari è comunque possibile stabilire a quale neonato corrisponde ciascun tracciato.
\item \column{TOCO}: misura la contrazione uterina, indicata come valore intero.
\item \column{FCM}: misura la frequenza cardiaca materna, indicata sempre come valore intero.
\end{itemize}

\subsection{Gynbase}
\label{gynbase}
\newcommand{\Gynbase}{\emph{Gynbase}}

Tra gli strumenti utilizzati dal personale medico del reparto figura il software \Gynbase{}, che si occupa di gestire i dati relativi alla gravidanza e al parto.
\Gynbase{} è un sistema proprietario che opera su una base di dati relazionale accessibile attraverso un'interfaccia grafica dedicata.
Tale programma è predisposto per contenere categorie di dati che non sono direttamente di interesse per la Clinica, per cui non sempre viene compilato quando i dati sono già presenti nelle tabelle esposte precedentemente (Sezione \ref{firsttrimester}, Sezione \ref{secondtrimester} e Sezione \ref{delivery}).
Il contenuto della base di dati, se esportato come tabella in formato CSV, arriva a comprendere circa 150 colonne; analizzeremo quindi gli attributi riportati come di interesse per il personale medico e che non si sovrappongono ad alti già esposti.

\begin{itemize}
\item \column{ID Paziente}: identificativo numerico progressivo attribuito dal sistema alla paziente.
\item \column{ID Gravidanza}: identificativo numerico progressivo attribuito dal sistema alla gravidanza. Questo numero è unico per l'intera base di dati.
\item \column{Cognome}, \column{Nome}, \column{Nata}: dati personali della paziente.
\item \column{Altezza cm}: altezza della paziente, come valore intero in centrimetri.
\item \column{Kg pre gravidanza}: peso della paziente, riportato da lei oppure misurato alla visita del primo trimestre.
\item \column{BMI pre gravidanza}: indice di massa corporea calcolato a partire da peso e altezza all'inizio della gravidanza.
\item \column{Età al concepimento}: indica l'età alla data stimata di concepimento, riportata come numero intero.
\item \column{UM}: riporta la data dell'ultima mestruazione.
\item \column{EPP}: riporta la data prevista del parto.
\item \column{PIH}: indica\footnote{
    I valori attesi della colonna \column{PIH}, così come delle seguenti \column{GDM}, \column{Tireopatia} e di altre colonne, sarebbero valori booleani del tipo Sì/No o 0/1. In realtà nel file CSV generato dal sistema viene riportato l'identificativo della paziente per significare \enquote{vero}/\enquote{sì} mentre invece è lasciato vuoto per significare \enquote{falso}/\enquote{no}. Il risultato è che i dati, in questo formato, non sono facilmente leggibili dall'utente.
} se la paziente ha sofferto di ipertensione gestazionale (\emph{pregnancy-induced hypertension}).
\item \column{GDM}: indica se la paziente ha sofferto di diabete gestazionale (\emph{gestational diabetes mellitus}).
\item \column{Tireopatia}: indica se la paziente ha sofferto di tiropatia.
\item \column{Data Espulsione}: indica data e ora\footnote{
    Data e ora vengono indicate in formato \texttt{DD/MM/YYYY HH:MM}, oppure con valore \enquote{date missing} se assente.
} del parto.
\item \column{Data Fase Attiva}: indica data e ora dell'inizio della fase attiva.
\item \column{Data Membrane}: indica data e ora della rottura delle membrane.
\item \column{Data Completa}: indica data e ora del raggiungimento della dilatazione completa.
\item \column{Data Fase Attiva 2}: indica data e ora dell'inizio della fase di espulsione o di spinta.
\item \column{Data Secondamento}: indica data e ora di espulsione della placenta.
\item \column{Complicazioni}: descrizione testuale di eventuali complicazioni avvenute nel processo di travaglio e parto.
\item \column{Indicazioni operativo}: dettagli sullo svolgimento del parto, se operativo.
\item \column{Ind ?}: indicazione del motivo o dei motivi dell'induzione del travaglio. Sono presenti diverse colonne, ciascuna rappresentativa di una motivazione (al posto di \column{?}): \column{ipertensione preesistente}, \column{ipertensione gestazionale}, \column{proteinuria isolata}, \column{iugr}, \column{gdm}, \column{preeclampsia}, \column{macrosomia}, \column{protratta}, \column{pma}, \column{oligoamnios}, \column{poliidramnios}, \column{altro}, \column{pprom}, \column{prom}, \column{colestasi}, \column{mef}, \column{prodromi prolungati}. Per indicare che l'induzione è stata svolta per uno o più determinati motivi la colonna riporta la lettera Y, altrimenti è lasciata vuota.
\item \column{Minuti induzione-inizio travaglio}: tempo trascorso tra l'induzione del travaglio e l'inizio del travaglio, indicato come valore intero.
\item \column{Minuti induzione-parto}: tempo trascorso tra l'induzione del travaglio e l'espulsione, indicato come valore intero.
\item \column{Bishop}: indicatore relativo alla preparazione del collo dell'utero all'induzione del travaglio, riportato come valore intero su una scala da 0 a 16.
\end{itemize}

\subsection{Tabella per gravidanze a rischio}
\label{pregnanciesatrisk}

Un insieme più ristretto di pazienti viene seguito nel corso della gravidanza da un ambulatorio appossito, detto appunto \enquote{gravidanze a rischio}, in presenza di determinate patologie.

\begin{itemize}
\item \column{Nome}, \column{Cognome}, \column{DdN}: dati identificativi della paziente.
\item \column{Patologie}: diverse colonne che riportano patologie, una per colonna, sofferte dalla paziente e che sono trattate o sorvegliate da questo ambulatorio.
\item \column{Terapie seguite}: descrizione testuale delle terapie seguite. Può non essere indicata una terapia per tutte le patologie.
\end{itemize}

\section{Problemi riscontrati con le soluzioni attuali}

L'insieme di tabelle e fogli di calcolo utilizzato attualmente dai medici del reparto è essenzialmente creato su misura, ovvero sono pensati per contenere le informazioni di loro interesse perché essi stessi ne hanno definito lo schema.
Nonostante ciò, questi strumenti sono molto più semplici di una base di dati ben strutturata e questo ne evidenzia gli aspetti negativi.

L'uso di tabelle, rispetto al contenuto testuale dei referti, permette di organizzare almeno visivamente l'insieme di informazioni che si vogliono conservare per ciascuna paziente, ma non vengono imposti vincoli di correttezza o di coerenza dei dati effettivamente inseriti.
Le operazioni di accesso ai dati (a partire dai dati identificativi della paziente) sono veloci se si considera una singola visita ma se si vuole analizzare l'intero periodo della gravidanza è necessario leggere contemporaneamente più tabelle; sotto questo aspetto l'interfaccia grafica fornita dai diversi software per visualizzare fogli di calcolo non è ottimale per tali scopi.

Il personale medico riporta anche difficoltà e insoddisfazione nell'uso del software Gynbase che, nonostante appaia meglio strutturato dal punto di vista grafico e concettuale, in realtà presenta aspetti negativi che emergono nel corso degli anni di esperienza.
I punti di criticità più importanti sono:
\begin{itemize}
    \item la mancanza di controlli o avvisi su dati incompatibili (ad esempio sull'ordine cronologico di determinati momenti del parto);
    \item il fatto che le stesse informazioni possono essere richieste in aree diverse del programma e quindi vengono duplicate, portando a evidenti inconsistenze se non c'è un adeguato controllo da parte di chi si interfaccia con il sistema;
    \item la poca chiarezza dell'interfaccia, in particolare nella distinzione tra informazioni necessarie e opzionali.
\end{itemize}

\section{Requisiti funzionali del nuovo sistema}
\label{functionalrequirements}

La richiesta avanzata dal personale medico è di un software integrato e centralizzato che permetta la visualizzazione, in prima analisi, e la manipolazione, in una fase successiva, dei dati relativi a gravidanze e parti, come esposto precedentente (Sezione \ref{problem}).

Il sistema dovrà permettere di visualizzare, aggiungere, modificare ed eliminare le informazioni che attualmente sono contenute nei diversi supporti esposti nella Sezione \ref{usednow}.
In sintesi, i dati riguardano dettagli personali della paziente e di sue eventuali patologie; dettagli della gravidanza, del suo decorso e delle visite svolte; informazioni su travaglio, parto e parametri fisiologici dei neonati; esiti dei vari esami eseguiti nel corso della gravidanza.

Analizzando le tabelle attualmente in uso possiamo dedurre le informazioni di interesse che dovranno sicuramente essere rese accessibili dal sistema che realizziamo.
Questo è anche il momento più indicato per valutare l'aggiunta di requisiti che prima potevano non essere stati considerati o non venivano realizzati per la mancanza di supporti adeguati.
In particolare, oltre a quanto esposto nella Sezione \ref{usednow}, i medici vogliono poter memorizzare informazioni sui seguenti aspetti non trattati prima.
\begin{enumerate}
\item Dati relativi ai neonati, in particolare peso, altezza e circonferenza cranica, misurati alla nascita.
\item Visite ecografiche biometriche. Possono essere svolte più visite biometriche nel corso della gravidanza (tendenzialmente 1 o 2) e si raccolgono dati relativi ai seguenti indicatori:
    \begin{itemize}
    \item BPD (diametro biparietale);
    \item CC (circonferenza cranica);
    \item AC (circonferenza addominale);
    \item FL (lunghezza del femore);
    \item EFW (peso fetale stimato);
    \item PIAO (valutazione flussimetrica dell'arteria ombelicale);
    \item PIACM (valutazione flussimetrica dell'arteria cerebrale media fetale);
    \item valutazione del dotto venoso;
    \item stato della crescita del feto (crescita regolare, FGR oppure SGA).
    \end{itemize}
\item Esiti di esami vari, eseguiti anche al di fuori delle principali visite elencate nella Sezione \ref{problem}, di una lista eventualmente espandibile su necessità del personale medico. Gli esami danno esiti in forma perlopiù di indici numerici, comprendenti ma non limitati a: HB, TSH, AST/ALT, glicemia, curva da carico di glucosio, sierologie varie, gruppo sanguigno, test di Coombs indiretto.
\item Dettagli su problemi concomitanti, patologie e terapie seguite (anche al di fuori di quelle controllate nell'ambulatorio \enquote{gravidanze a rischio}) e altri aspetti come interventi, trasfusioni o allergie, da indicare in forma testuale o debolmente strutturata.
\end{enumerate}

Riguardo invece alle informazioni presenti nei sistemi già in uso (Sezione \ref{usednow}) vengono fatte alcune precisazioni:
\begin{itemize}
\item I parti naturali e operativi possono essere eseguiti o meno con la manovra di Kristeller.
\item Un parto gemellare può avere delle complicazioni tali da portare un gemello a nascere naturalmente e l'altro con un parto cesareo oppure operativo. In questi casi si vogliono mantenere informazioni sugli aspetti caratteristici di entrambi i tipi di parto avvenuti.
\item Durante il travaglio possono essere somministrati (se farmacologici) o eseguiti (se meccanici) diversi metodi di induzione. Le informazioni su metodo, tempo di somministrazione, motivo, valutazione di Bishop, completamento della procedura, numero di cicli eseguiti e quantità somministrata si vogliono mantenere per ciascuna somministrazione.
\end{itemize}

\section{Scelta di una base di dati di tipo relazionale}

\subsection{Modello relazionale}

Nel modello relazionale \cite{Cod70} i dati vengono strutturati attraverso relazioni (o equivalentemente, tabelle), ciascuna composta di attributi (colonne) e istanziata come insieme di tuple (righe o record).
Uno schema relazionale corretto e ben strutturato mira a prevenire inconsistenze e dati duplicati.
Gli strumenti software che operano su basi di dati relazionali (RDBMS) permettono di definire vincoli di integrità che contribuiscono al mantenimento della coerenza dei dati, all'assenza di ambiguità e all'efficienza delle interazioni con il sistema attraverso controlli automatici.
La modellazione del dominio in cui operiamo viene fatta prima a livello concettuale definendo uno schema nel modello \enquote{Entità-Relazione} e successivamente traducendolo in uno schema logico nel modello relazionale.

Le basi di dati relazionali si possono interrogare con semplici richieste (\emph{query}) in SQL, un linguaggio dichiarativo basato sulla logica del prim'ordine.
Questo linguaggio è supportato da tutti i principali linguaggi di programmazione e ambienti di sviluppo, prestandosi così a integrarsi facilmente in applicazioni complesse e che permettono, ad esempio, l'interazione attraverso un'interfaccia grafica dedicata.
In questo modo solo il programmatore ha la necessità di conoscere i dettagli implementativi dello schema mentre l'utente finale, anche se inesperto, può operare sui dati a livello più astratto e vicino al modello concettuale del dominio.

\subsection{Confronto con altri modelli}

Altri modelli di basi di dati possono avere altri vantaggi specifici, ma spesso perdono su alcuni dei punti forti del modello relazionale.
\begin{itemize}
\item Il modello gerarchico prevede una struttura ad albero, composto da nodi che possono avere un numero arbitrario di figli ma solo un genitore ciascuno. Questo pone molti limiti sullo schema da realizzare e spesso richiede di duplicare dati.
\item Il modello reticolare estende quello gerarchico permettendo un numero arbitrario di genitori, formando quindi un grafo generico. La struttura così ottenuta, eccessivamente libera nella sua topologia, rende più difficoltosa l'interrogazione.
\item Il modello a documento, che sta prendendo piede come principale modello di tipo \emph{NoSQL}, prevede che i record siano memorizzati come documenti strutturati ma senza uno schema comune. Questo facilita la memorizzazione di dati con caratteristiche eterogenee ma allo stesso tempo rende più difficoltose le operazioni che legano documenti diversi, come le \emph{join}.
\end{itemize}

\subsection{Applicazione al caso di studio}

Considerando i requisiti esposti nella Sezione \ref{functionalrequirements}, proponiamo l'utilizzo di una base di dati basata su un modello relazionale.
Le informazioni trattate dal personale medico seguono schemi ben strutturati, come dimostra l'organizzazione tabellare utilizzata finora, e a tali schemi possiamo ricondurre le entità e le relazioni a livello astratto.
Riconosciamo infatti entità come la paziente, la gravidanza, la visita, il parto, il neonato, ciascuna con un insieme di attributi.
Tra di esse sussistono relazioni che seguono schemi costanti, come la relazione tra gravidanze e visite, tra gravidanze e malattie o tra parti e neonati.

Una base di dati relazionale permette di gestire in modo naturale diversi vincoli di integrità, soprattutto per quanto riguarda gli attributi identificativi (detti \enquote{chiavi}) sia il numero di entità coinvolte in determinate relazioni (detto \enquote{cardinalità}).
Nel caso reale che stiamo trattando è possibile porre vincoli sul numero di visite di ciascun tipo effettuate nel corso della gravidanza, così come sul numero di bambini nati da un parto o di esami svolti in ciascuna visita.

Si vuole che il sistema risultante costituisca, in prima fase, un magazzino dati (\emph{data warehouse}\footnote{
    L'accezione di \emph{data warehouse} \cite{Ded16}, concetto distinto da \emph{database}, evidenzia come il sistema integri dati provenienti da più fonti in un unico schema e sia orientato alla visualizzazione più che alle operazioni di modifica.
}) volto principalmente alla visualizzazione contemporanea e coerente dei dati proveniente dalle diverse fonti attualmente in uso e che possa sostituirle completamente in un secondo momento.

Nel Capitolo \ref{conceptual} si mostra lo schema Entità-Relazione costruito a partire dai requisiti e vengono illustrate le scelte operate a livello concettuale, con tutti i dettagli relativi alle entità presenti, alle relazioni tra esse e agli attributi.