%% Capitolo
\chapter{Analisi del problema e dei requisiti}

Lo sviluppo di una base di dati centralizzata per la Clinica Ostetrica dell'Ospedale di Udine si inserisce in un progetto di studio più ampio, mirato alla raccolta e all'analisi dei dati relativi a gravidanze e parti, al fine di migliorare l'assistenza fornita dal personale medico e la conseguente esperienza delle pazienti.
Nell'ambito medico in generale la raccolta dati investe un ruolo cruciale: attraverso l'analisi e l'interpretazione dei risultati statistici è possibile in particolare:
\begin{itemize}
\item migliorare la cura del paziente individuando modelli, tendenze e fattori di rischio, consentendo così ai medici di predisporre strumenti di prevenzione adatti \cite{Cor20,McC16};
\item promuovere la ricerca scientifica stimolando nuovi studi a partire dai risultati, potenzialmente inediti, estrapolati dalla mole di dati trattata dal personale ospedaliero;
\item ottimizzare la gestione ospedaliera grazie alle informazioni su efficienza operativa, uso delle risorse e conseguenti risultati.
\end{itemize}
Più nello specifico, la Clinica Ostetrica presso l'Ospedale \enquote{Santa Maria della Misericordia} di Udine è interessata a sostituire i diversi strumenti software attualmente in uso con un sistema centralizzato, personalizzato secondo i requisiti dei medici e che migliori l'accesso ai dati sia per la sintesi dei dati della singola paziente sia per analisi statistiche di larga scala.

\section{Analisi dei supporti software in uso}

Gli strumenti di raccolta dati che il personale medico utilizza attualmente sono un insieme eterogeneo. I dati sono distribuiti tra software diversi e con un livello di strutturazione molto variabile, da fogli di calcolo compilati manualmente fino a basi di dati relazionali a cui si accede attraverso interfacce grafiche.

I dati vengono raccolti in momenti diversi nel corso della gravidanza, che possiamo ricondurre principalmente a: una visita che si effettua al primo trimestre, una visita al secondo trimestre, il momento del parto.