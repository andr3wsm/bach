\begin{dedication}
Alle donne di Gaza,
\par che nella tragedia di oltre 20 mesi di massacri e sofferenze sono costrette a subire la mancanza di cure mediche e assistenza durante la gravidanza, e specialmente al momento del parto, spesso non potendo disporre adeguatamente di cibo, medicinali e anestesia.
\end{dedication}

\acknowledgements
Nella realizzazione di questo progetto è stato fondamentale il contributo del personale della Clinica Ostetricia~-~Ginecologia dell'Ospedale di Udine, sia sul lato dell'acquisizione dei requisiti strettamente necessari per progettare la base di dati sia per quanto riguarda la conoscenza del tema della salute in gravidanza e per i legami che ho potuto scoprire tra la medicina e l'informatica.
Vorrei ringraziare quindi prima di tutto la professoressa Driul e le dottoresse Armenise, Battello, Rizzante, Xodo e Zermano, sperando di aver contribuito in modo positivo alle necessità del loro reparto con la costruzione della base di dati.

Ringrazio il professor Montanari e i dottori Brunello e Saccomanno per la disponibilità e la motivazione che mi hanno dato nel portare avanti questo progetto.
Insieme a loro ringrazio anche la mia famiglia, la mia fidanzanta Martina, i compagni del corso di Informatica e della Scuola Superiore, per il sostegno che ho sempre ricevuto da parte loro e per il piacere di aver condiviso questo percorso.

\abstract
In questa tesi esporremo la progettazione e l'implementazione di una base di dati relazionale, costruita secondo i requisiti ricavati direttamente dal personale medico, per informazioni su gravidanze e parti nel reparto di Ginecologia e Ostetricia nell'Ospedale di Udine, con l'obiettivo di integrare un sistema che sia adatto alle esigenze degli utenti e allo stesso tempo privo di inconsistenze nei dati.

Analizzeremo gli strumenti attualmente in uso, in particolare le loro criticità, e le richieste aggiuntive avanzate dai medici. In seguito, costruiremo uno schema concettuale per rappresentare il dominio del sistema. Attraverso una fase di ristrutturazione, defineremo uno schema logico e successivamente implementeremo tabelle e vincoli di integrità attraverso \emph{script} SQL. Infine, porteremo esempi di interrogazioni possibili su tale base di dati.