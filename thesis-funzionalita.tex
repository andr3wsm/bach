\chapter{Funzionalità della base di dati}
\label{functionality}

Come anticipato nella Sezione \ref{choice}, una base di dati relazionale permette di memorizzare dati in maniera strutturata ed efficiente e allo stesso tempo mantiene i vincoli imposti sui dati che prevengono le inconsistenze.
Vogliamo mostrare e valutare le qualità del sistema ottenuto e confrontarlo con la soluzione attualmente in uso; ciò viene fatto principalmente con l'esemplificazione di \emph{query} per l'interrogazione della base di dati.

\section{Esempi di query} % provvisorio

Generalmente il personale medico è interessato a inserire, visualizzare e modificare le informazioni relative a singole pazienti e raccolte in singoli momenti, ad esempio l'inserimento in blocco dei dati di un parto o di una visita.
Queste operazioni, che risultano immediate nella soluzione attuale che impiega fogli di calcolo strutturati proprio secondo i diversi eventi della gravidanza, rimangono non solo possibili ma soprattutto facilmente esprimibili in SQL.
Questo avviene perché, nonostante le ristrutturazioni e gli adattamenti necessari all'implementazione, le tabelle dello schema logico e fisico sono ancora perlopiù corrispondenti a entità e relazioni dello schema concettuale.

Ci concentriamo quindi su un livello diverso di analisi dei dati, ovvero principalmente lo studio di dati aggregati e la selezione di record sulla base di caratteristiche di interesse per gli utilizzatori finali.
Portiamo alcuni esempi di interrogazioni richieste dai medici che, grazie alla struttura tabellare e alle funzionalità dei DBMS relazionali, vengono svolte efficientemente e senza che sia richiesta una processazione manuale aggiuntiva.
Gli esempi di \emph{query} mostrati di seguito non hanno l'obiettivo di portare a un'effettiva analisi delle informazioni presenti nell'istanza di base di dati, bensì hanno il solo scopo di mostrare come tali informazioni possono essere estratte.

\subsubsection{Valori di BMI e lacerazioni}

Si vuole ottenere, per ogni gravidanza, il valore del BMI registrato all'inizio della gravidanza e il grado di lacerazioni subito nel parto.
Questo risultato è ottenibile con l'interrogazione del \lstlistingname{} \ref{qbmilacerazioni}.

\subsubsection{Correlazione tra taglio cesareo e travaglio indotto}

Si vuole ottenere la percentale di tagli cesarei effettuati nei parti con travaglio indotto confrontata con quella nei parti con travaglio spontaneo.
Questo risultato è ottenibile con l'interrogazione del \lstlistingname{} \ref{qcesareotravaglio}.
Definiamo una vista \tab{parto\_con\_travaglio\_induzione} per rendere più concise le diverse \emph{subquery} nel corpo della \emph{query} principale.

\subsubsection{Correlazione tra parto operativo ed episiotomia}

Si vuole ottenere la percentuale di episiotomie effettate nei parti operativi rispetto ai parti vaginali.
Questo risultato è ottenibile con l'interrogazione del \lstlistingname{} \ref{qoperativoepisiotomia}, che risulta essere molto simile al primo esempio di \emph{query}.

\subsubsection{Correlazione tra taglio cesareo e indice Robson}

Si vuole ottenere il numero di tagli cesarei e di parti vaginali per ciascun valore dell'indice Robson.
Questo risultato è ottenibile con l'interrogazione del \lstlistingname{} \ref{qcesareorobson}; definiamo e usiamo una vista \tab{parto\_sottotipo} per rendere più concisa la \emph{query} principale.

Per evitare doppi conteggi, questa interrogazione non considera i parti che non sono stati soltanto vaginali o soltanto cesarei.
Ricordiamo che è possibile avere una sovrapposizione nel sottotipo dei parti con travaglio nel caso di parti gemellari in cui i gemelli nascono con procedure diverse.

\begin{lstlisting}[float,caption={Esempio di \emph{query}. Valori di BMI e lacerazioni.},label=qbmilacerazioni]
-- Query principale
select *
from
  (select gravidanza_id, peso/(altezza*altezza) as bmi
  from visita
  where categoria_visita = 'primo_trimestre'
    and altezza is not null
    and altezza <> 0
    and peso is not null) b
  natural join
  (select gravidanza_id, lacerazioni
  from parto_con_travaglio);
\end{lstlisting}

\begin{lstlisting}[float,caption={Esempio di \emph{query}. Correlazione tra parto cesareo e travaglio indotto.},label=qcesareotravaglio]
-- Definizione di una vista con il numero di induzioni per parto
create view parto_con_travaglio_induzione as
select *
from parto_con_travaglio natural join
  (select count(*) as induzioni, gravidanza_id
  from induzione
  group by gravidanza_id);
-- Query principale
select s1.tc_travaglio_spontaneo as tc_travaglio_spontaneo,
  s2.parti_travaglio_spontaneo as parti_travaglio_spontaneo,
  case when s2.parti_travaglio_spontaneo = 0
    then 0
	  else s1.tc_travaglio_spontaneo::numeric/s2.parti_travaglio_spontaneo
  end as rapporto_tc_travaglio_spontaneo,
  s3.tc_travaglio_indotto as tc_travaglio_indotto,
  s4.parti_travaglio_indotto as parti_travaglio_indotto,
  case when s4.parti_travaglio_indotto = 0
    then 0
	  else s3.tc_travaglio_indotto::numeric/s4.parti_travaglio_indotto
  end as rapporto_tc_travaglio_indotto
from
-- Numero di tagli cesarei in travaglio spontaneo
(select count(*) as tc_travaglio_spontaneo
from parto_con_travaglio_induzione
where induzioni = 0
  and sottotipo_parto like '%cesareo%') s1,
-- Numero di parti totali in travaglio spontaneo
(select count(*) as parti_travaglio_spontaneo
from parto_con_travaglio_induzione
where induzioni = 0) s2,
-- Numero di tagli cesarei in travaglio indotto
(select count(*) as tc_travaglio_indotto
from parto_con_travaglio_induzione
where induzioni > 0
  and sottotipo_parto like '%cesareo') s3,
-- Numero di parti totali in travaglio indotto
(select count(*) as parti_travaglio_indotto
from parto_con_travaglio_induzione
where induzioni > 0) s4;
\end{lstlisting}

\begin{lstlisting}[float,caption={Esempio di \emph{query}. Correlazione tra parto operativo ed episiotomia.},label=qoperativoepisiotomia]
-- Query principale
select s1.ep_operativi as episiotomie_operativi,
  s2.parti_operativi as parti_operativi,
  case when s2.parti_operativi = 0
    then 0
	  else s1.ep_operativi::numeric/s2.parti_operativi
  end as rapporto_episiotomie_operativi,
  s3.ep_vaginali as episiotomie_vaginali,
  s4.parti_vaginali as parti_vaginali,
  case when s4.parti_vaginali = 0
    then 0
	  else s3.ep_vaginali::numeric/s4.parti_vaginali
  end as rapporto_episiotomie_vaginali
from
-- Numero di tagli cesarei in travaglio spontaneo
(select count(*) as ep_operativi
from parto_con_travaglio
where lacerazioni = 'episiotomia'
  and sottotipo_parto = 'operativo') s1,
-- Numero di parti totali in travaglio spontaneo
(select count(*) as parti_operativi
from parto_con_travaglio
where sottotipo_parto = 'operativo') s2,
-- Numero di tagli cesarei in travaglio indotto
(select count(*) as ep_vaginali
from parto_con_travaglio
where lacerazioni = 'episiotomia'
  and sottotipo_parto = 'naturale') s3,
-- Numero di parti totali in travaglio indotto
(select count(*) as parti_vaginali
from parto_con_travaglio
where sottotipo_parto = 'naturale') s4;
\end{lstlisting}

\begin{lstlisting}[float,caption={Esempio di \emph{query}. Correlazione tra parto cesareo e indice Robson.},label=qcesareorobson]
-- Definizione di una vista con il sottotipo di ogni parto
create view parto_sottotipo as
select
  gravidanza_id,
  robson,
  case when tipo_parto = 'cesareo_programmato'
    then 'cesareo'
    else sottotipo_parto
  end as sottotipo_parto
from parto natural left join parto_con_travaglio;
-- Query principale
select robson, parto_sottotipo as tipo, count(*) as numero_parti
from parto_sottotipo
where sottotipo_parto in ('cesareo','naturale')
group by robson, sottotipo_parto
order by robson, sottotipo_parto;
\end{lstlisting}