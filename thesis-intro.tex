\chapter{Introduzione}

Lo sviluppo di una base di dati centralizzata per la Clinica Ostetrica dell'Ospedale di Udine si inserisce in un progetto di studio più ampio, mirato alla raccolta e all'analisi dei dati relativi a gravidanze e parti, al fine di migliorare l'assistenza fornita dal personale medico e la conseguente esperienza delle pazienti.
Nell'ambito medico in generale la raccolta dati investe un ruolo cruciale: attraverso l'analisi e l'interpretazione dei risultati statistici è possibile in particolare:
\begin{itemize}
\item migliorare la cura del paziente individuando modelli, tendenze e fattori di rischio, consentendo così ai medici di predisporre strumenti di prevenzione adatti \cite{Cor20,McC16};
\item promuovere la ricerca scientifica stimolando nuovi studi a partire dai risultati, potenzialmente inediti, estrapolati dalla mole di dati trattata dal personale ospedaliero;
\item ottimizzare la gestione ospedaliera grazie alle informazioni su efficienza operativa, uso delle risorse e conseguenti risultati.
\end{itemize}
Più nello specifico, la Clinica Ostetrica presso l'Ospedale \enquote{Santa Maria della Misericordia} di Udine è interessata a sostituire i diversi strumenti software attualmente in uso con un sistema centralizzato, personalizzato secondo i requisiti dei medici e che migliori l'accesso ai dati sia per la sintesi dei dati della singola paziente sia per analisi statistiche di più ampia scala.

\section{Obiettivo del progetto}

Come verrà esposto nella Sezione \ref{problem}, gli usuali sistemi di refertazione adottati a livello aziendale risultano inadatti per la visualizzazione sintetica e le analisi sui dati.
I referti sono costituiti di solo testo, oltre a mancare di una struttura logica; ciò ha portato alla scelta di registrare i dati in documenti separati, principalmente in forma di tabelle compilate a mano a seguito delle visite.

Questo sistema parallelo, costruito progressivamente secondo le esigenze specifiche dei suoi utenti, è in uso da circa x anni.
I medici hanno evidenziato delle problematiche nel suo utilizzo, relative principalmente alla velocità nell'acceso alle informazioni e all'insorgenza di ambiguità legate alla compilazione esclusivamente manuale e da parte di persone diverse.

Al fine di fornire uno strumento informatico che sia più funzionale, accessibile e coerente proponiamo lo sviluppo e l'implementazione di una base di dati di tipo relazionale che sia modellata sulle esigenze specifiche del contesto in cui verrà utilizzata.

\section{Realizzazione del progetto}

Il processo comprende una prima fase di analisi dei requisiti, volta a individuare le caratteristiche richieste dagli utenti finali e dagli \emph{stakeholder} in generale, seguita da una fase di modellazione a livello concettuale che determina le caratteristiche del dominio di interesse.
A partire dal modello concettuale del dominio costruiamo uno schema logico relazionale che permette la definizione di operazioni di base sui dati, come interrogazioni e inserimenti.
L'ultima fase consiste nell'implementazione con software specifici che permettono l'interazione attraverso il linguaggio SQL.
Mostreremo quindi un insieme di analisi di dati ottenibili grazie a interrogazioni sulla base di dati, per evidenziare in particolar modo le azioni rese possibili dal nuovo sistema che non erano possibili o praticabili con gli strumenti precedenti.

% completare con anticipazione delle conclusioni