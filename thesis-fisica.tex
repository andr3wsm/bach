\chapter{Progettazione fisica}

La fase di progettazione logica (Sezione \ref{logical}) produce uno schema logico, formalizzato nel modello relazionale, che descrive l'insieme di tabelle e di vincoli su di esse che contengono i dati di interesse.
Questo schema è interrogabile, ovvero è possibile formulare delle interrogazioni (anche dette \emph{query} o richieste) sulla base di dati; le interrogazioni si formulano attraverso linguaggi formali specifici per il modello relazionale come algebra relazionale e calcolo relazionale.

Il linguaggio SQL (\emph{Structured Query Language}) è un linguaggio di interrogazione per basi di dati relazionali \cite{Sil11}, affermatosi come standard \emph{de facto} (grazie alle numerose implementazioni nei sistemi distribuiti commercialmente) e anche \emph{de iure}\footnote{
Il primo standard ANSI e ISO per il linguaggio SQL è del 1986 (SQL-86) \cite{Sil11}. Successivamente sono stati pubblicati diversi aggiornamenti fino alla versione attuale, SQL:2023 (ISO/IEC 9075:2023).
}.
SQL è composto di diverse parti: è possibile esprimere sia la definizione delle tabelle (DDL, \emph{Data Definition Language}) sia le operazioni di inserimento, modifica, cancellazione e interrogazione (DML, \emph{Data Manipulation Language}), insieme anche ai vincoli di integrità e ai privilegi degli utenti che possono accedere alla base di dati.
In questo capitolo costruiamo fisicamente la base di dati progettata nei capitoli precedenti attraverso diversi \emph{script} in linguaggio SQL.


\section{Definizione dei domini degli attributi}

\section{Definizione delle tabelle}

\section{Definizione dei vincoli}

\section{Definizione dei trigger}