\chapter{Progettazione concettuale}

\section{Scelte di modellazione}

L'analisi dei requisiti e dei supporti esistenti (Sezione \ref{usednow}) evidenzia come elemento centrale le gravidanze, i cui dati vengono registrati nel corso di diversi momenti.
Solo poche informazioni sono legate alla paziente piuttosto che alla gravidanza e sono essenzialmente quelle che la identificano personalmente.
Si introduce un identificativo per la paziente che risulta essere effettivamente univoco, ovvero il codice fiscale, assumendo che tutte le pazienti trattate abbiano un codice fiscale assegnato; ciò semplifica la rappresentazione dell'identificativo personale, che nello schema logico verrà inserito in quasi tutte le tabelle, rispetto alla tripla composta da nome, cognome e data di nascita, e previene i possibili, seppur rarissimi, casi di corrispondenza di questi dati per persone diverse.

\begin{figure}
\includesvg{vectors/modello-er-originale.svg}
\caption{Schema Entità-Relazione semplificato contenente soltanto entità e relazioni del modello. Le aree tratteggiate delimitano i diversi ingrandimenti nei quali vengono specificati gli attributi delle entità e relazioni coinvolte.}
\label{completeerdiagram}
\end{figure}

\section{Schema Entità-Relazione}