\chapter{Progettazione concettuale}
\label{conceptual}

\newcommand{\ent}[1]{{\Large #1}}
\newcommand{\card}[1]{{\footnotesize #1}}
\newcommand{\CZU}{\card{(0,1)}}
\newcommand{\CZN}{\card{(0,n)}}
\newcommand{\CUU}{\card{(1,1)}}
\newcommand{\CUN}{\card{(1,n)}}

\begin{figure}
    \centering
    \includesvg{vectors/modello-er-originale.svg}
    \caption{Schema Entità-Relazione. Visione d'insieme semplificata, contenente soltanto entità e relazioni del modello. Le aree tratteggiate delimitano i diversi ingrandimenti nei quali vengono specificati gli attributi delle entità e relazioni coinvolte.}
    \label{completeerdiagram}
\end{figure}

La figura \ref{completeerdiagram} mostra lo schema della base di dati nel modello Entità Relazione in una versione semplificata per evidenziare le entità e le relazioni presenti.
I quattro contorni tratteggiati identificano quattro aree del diagramma (gravidanza, visite, parto, neonato) che verranno illustrate dettagliatamente con gli attributi presenti.
L'analisi dei requisiti e dei supporti esistenti (Sezione \ref{usednow}) evidenzia come elemento centrale le gravidanze: solo poche informazioni sono legate alla paziente piuttosto che alla gravidanza e sono essenzialmente quelle che la identificano personalmente.

In generale la presenza di attributi opzionali è dovuta non tanto alla mancanza del dato, ovvero che non sia presente a livello concettuale, bensì alla mancata conoscenza di un suo valore, da attribuirsi a un'errata compilazione dei record o a un dettaglio anamnestico non riportato dalla paziente.
Nell'uso del valore \emph{NULL} si ritrova questa stessa ambiguità \cite{Sil11}.
Si intende lasciare la possibilità di avere dati mancanti per permettere un'integrazione che sia compatibile con i sistemi attualmente in uso, in modo da implementare il sistema come \emph{data warehouse}, oltre a renderlo tollerante rispetto a eventuali omissioni di informazioni.

\section{Area concettuale della gravidanza}
\label{pregnancyconceptual}

\begin{figure}
    \centering
    \includesvg{vectors/modello-er-originale-gravidanza.svg}
    \caption{Schema Entità Relazione. Ingrandimento dell'area gravidanza.}
    \label{pregnancyerdiagram}
\end{figure}

La Figura \ref{pregnancyerdiagram} mostra l'ingrandimento dello schema Entità-Relazione per quanto riguarda le entità Paziente, Gravidanza e Malattia.

\subsubsection{Paziente}

Si introduce un identificativo per la paziente che risulta essere effettivamente univoco, ovvero il codice fiscale, assumendo che tutte le pazienti trattate abbiano un codice fiscale assegnato.
Ciò semplifica la rappresentazione dell'identificativo personale, che nello schema logico verrà inserito in quasi tutte le tabelle, rispetto alla tripla composta da nome, cognome e data di nascita, e previene i possibili, seppur rarissimi, casi di corrispondenza di questi dati per persone diverse.

\subsubsection{Gravidanza}

A questa entità fanno riferimento alcuni attributi che vengono registrati durante la visita del primo trimestre, oltre all'esito della gravidanza e alla data del primo ingresso.
Decidere se assegnare tali attributi a \enquote{Gravidanza} oppure a \enquote{Visita I° trimestre} può essere talvolta arbitrario: preferiamo quindi assegnare alla prima quelle informazioni che si possono considerare come proprie della gravidanza in sé (come la presenza di feti gemellari o il tipo di PMA), mentre alla seconda le informazioni sulla paziente che i si riferiscono al giorno o al periodo della visita (come ad esempio peso o altezza).

Identifichiamo due chiavi candidate, entrambe parziali in quanto l'entità è debole verso Paziente:
\begin{itemize}
\item la data del primo ingresso, attributo derivato e corrispondente alla data della prima visita effettuata, se ce n'è stata una, altrimenti alla data del parto;
\item la parità registrata all'inizio della gravidanza.
\end{itemize}

L'attributo \enquote{Parità} è composto.
Nei sistemi attualmente in uso è scritto solitamente in un unico campo, usando quindi la rappresentazione testuale che è calcolata come concatenazione del numero di figli nati a termine, del numero di figli nati pretermine, del numero di aborti (comprendente sia aborti spontanei sia interruzioni volontarie di gravidanza) e del numero totale di figli nati vivi, in questo ordine.

Sempre sull'attributo \enquote{Parità} si pone il seguente vincolo.
\begin{quote}
Se una paziente ha più di una gravidanza, per ogni coppia di gravidanze successive i valori degli attributi (non derivati) sono non decrescenti e almeno uno di essi è strettamente crescente.
\end{quote}

Le relazioni che l'entità Gravidanza intrattiene con le diverse visite e con Parto hanno tutte partecipazione parziale, per cui viene imposto il seguente vincolo.
\begin{quote}
Una gravidanza viene registrata solo se viene effettuata almeno una visita oppure se termina con un parto.
\end{quote}

\subsubsection{Malattia}

L'entità \enquote{Malattia} rappresenta le patologie che possono presentarsi in concomitanza con la gravidanza.
Attualmente vengono registrate in modo sistematico solo per le pazienti seguite nell'ambulatorio \enquote{gravidanze a rischio} (vedi Sezione \ref{pregnanciesatrisk}) ma è applicabile alla gravidanza di qualsiasi paziente.
È importante considerare che le malattie possono variare tra le diverse gravidanze della stessa paziente.
Le informazioni riguardo alle eventuali terapie seguite sono memorizzate nell'attributo della relazione che lega gravidanza e malattie.

\section{Area concettuale delle visite}
\label{visitsconceptual}

\begin{figure}
    \centering
    \includesvg{vectors/modello-er-originale-visite.svg}
    \caption{Schema Entità Relazione. Ingrandimento dell'area visite.}
    \label{visitserdiagram}
\end{figure}

La Figura \ref{visitserdiagram} mostra l'ingrandimento dello schema Entità-Relazione per quanto riguarda le entità relative alle visite e all'entità Esame.

\subsubsection{Visite}

Dividiamo le possibili visite in quattro entità: Visita I° trimestre, Visita II° trimestre, Visita biometrica, Visita di altro tipo.
Gli attributi di queste entità fanno riferimento principalmente alla data e alle informazioni rilevate dalla paziente come età, epoca gestazionale o anomalie fetali.
Gli esiti degli esami svolti per una determinata visita non sono modellati come attributi delle visite ma come attributi della relazione che lega Visita e Esame, in modo che gli esami da registrare possano essere modificati nel tempo senza agire sullo schema della base di dati.

I diversi attributi \enquote{Età}, presenti in tutte le entità, sono calcolati a partire dalla data di nascita della paziente.
L'entità \enquote{Visita di altro tipo} raccoglie le istanze di visite che altrimenti non rientrerebbero in nessuna delle tre categorie specifiche ma per le quali si vogliono comunque registrare esiti di esami svolti.

La scelta di organizzare l'entità \enquote{Visita} come unione (o categoria) piuttosto che come una generalizzazione è dovuta alla cardinalità delle relazioni che i quattro tipi di visita intrattengono con l'entità Gravidanza.
Tutte le visite sono entità deboli rispetto a Gravidanza, ma le visite biometriche o di altro tipo possono essere svolte più volte nel corso di una gravidanza e quindi richiedono un attributo che sia chiave parziale, che invece sarebbe scorretto mettere per le visite del primo e del secondo trimestre le quali sono effettuate una sola volta ciascuna.
Se fossero state modellate come specializzazioni di un'unica entità Visita si sarebbero creati numerosi cicli e vincoli aggiuntivi da imporre. 

\subsubsection{Esame}

L'entità Esame rappresenta l'esame in generale, mentre l'esito di uno specifico esame svolto si ritrova nella relazione tra Esame e Visita.
Il tipo di dato dell'esito dipende dall'esame, ma l'aspetto implementativo viene trattato a livello di schema logico.

\section{Area concettuale del parto}
\label{deliveryconceptual}

\begin{figure}
    \centering
    \includesvg{vectors/modello-er-originale-parto.svg}
    \caption{Schema Entità Relazione. Ingrandimento dell'area parto.}
    \label{deliveryerdiagram}
\end{figure}

La Figura \ref{deliveryerdiagram} mostra l'ingrandimento dello schema Entità-Relazione per quanto riguarda le entità Parto, con le sue specializzazioni, e Travaglio.

\subsubsection{Parto}

Le diverse specializzazioni dell'entità Parto sono totali e disgiunte.
Solo alcune informazioni, come data o informazioni su analgesia e perdite ematiche, sono applicabili ad ogni tipo di parto: gli altri attributi si trovano nelle entità che specializzano Parto, in alcuni casi con sovrapposizioni.
Ad esempio, i dati sul travaglio e sulle tempistiche del parto sono applicabili a tutti i tipi di parto eccetto i parti cesarei programmati, mentre le lacerazioni e il secondamento sono dati presenti nei parti naturali e operativi.
La data del parto è coerente con le tempistiche indicate se il parto non è cesareo urgente.

\subsubsection{Travaglio}

Le informazioni sul travaglio sono applicabili a tutti e soli i parti che non sono cesarei programmati.
Se il travaglio è indotto sono presenti anche i dati relativi al metodo e alle tempistiche dell'induzione.

\section{Area concettuale del neonato}
\label{newbornconceptual}

\begin{figure}
    \centering
    \includesvg{vectors/modello-er-originale-neonato.svg}
    \caption{Schema Entità Relazione. Ingrandimento dell'area neonato.}
    \label{newbornerdiagram}
\end{figure}

La Figura \ref{newbornerdiagram} mostra l'ingrandimento dello schema Entità-Relazione per quanto riguarda le entità Neonato e quelle relative alle misurazioni del cardiotocografo.

\subsubsection{Neonato}

Nell'entità Neonato la chiave parziale è l'ora di nascita.
Questo, ugualmente ad altri indicatori dell'ordine di nascita, è necessario per distinguere i gemelli nati in una stessa gravidanza.

Le informazioni relative al neonato sono le seguenti:
\begin{itemize}
\item peso, altezza e sesso attribuito;
\item indice di Apgar, misurato in diversi momenti dopo la nascita;
\item BE: \emph{base excess}, eccesso di basi;
\item pH: acidità del sangue;
\item CC: circonferenza cranica;
\item TIN: terapia intensiva neonatale.
\end{itemize}

\subsubsection{Tracciato e misurazioni}

Nello schema la sigla CTG sta per \enquote{cardiotocografo}.
Un tracciato è identificabile sia con il parto in cui viene registrato sia con il numero progressivo, che viene assegnato unico dal sistema in uso nel reparto.

Ogni misurazione del tracciato riporta i valori FCM (frequenza cardiaca materna) e TOCO (tocogramma).
I valori FCF (frequenza cardiaca fetale) possono essere multipli in ogni misurazione, quindi non sono contenute in un attributo dell'entità Misurazione TCG bensì nella relazione tra essa e Neonato.
Si impone quindi il seguente vincolo.
\begin{quote}
Per ogni misurazione deve essere presente almeno uno tra: il valore FCM, il valore TOCO e un valore FCF per un neonato associato.
\end{quote}