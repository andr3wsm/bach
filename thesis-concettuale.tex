\chapter{Progettazione concettuale}
\label{conceptual}

\newcommand{\ent}[1]{{\Large #1}}
\newcommand{\card}[1]{{\footnotesize #1}}
\newcommand{\CZU}{\card{(0,1)}}
\newcommand{\CZN}{\card{(0,n)}}
\newcommand{\CUU}{\card{(1,1)}}
\newcommand{\CUN}{\card{(1,n)}}

\begin{figure}
    \centering
    \includesvg{vectors/modello-er-originale.svg}
    \caption{Schema Entità-Relazione. Visione d'insieme semplificata, contenente soltanto entità e relazioni del modello. Le aree tratteggiate delimitano i diversi ingrandimenti nei quali vengono specificati gli attributi delle entità e relazioni coinvolte.}
    \label{completeerdiagram}
\end{figure}

La figura \ref{completeerdiagram} mostra lo schema della base di dati nel modello Entità Relazione in una versione semplificata per evidenziare le entità e le relazioni presenti.
I quattro contorni tratteggiati identificano quattro aree del diagramma (gravidanza, visite, parto, neonato) che verranno illustrate dettagliatamente con gli attributi presenti.
L'analisi dei requisiti e dei supporti esistenti (Sezione \ref{usednow}) evidenzia come elemento centrale le gravidanze: solo poche informazioni sono legate alla paziente piuttosto che alla gravidanza e sono essenzialmente quelle che la identificano personalmente.

In generale la presenza di attributi opzionali è dovuta non tanto alla mancanza del dato, ovvero che non sia presente a livello concettuale, bensì alla mancata conoscenza di un suo valore, da attribuirsi a un'errata compilazione dei record o a un dettaglio anamnestico non riportato dalla paziente.
Nell'uso del valore \emph{NULL} si ritrova questa stessa ambiguità \cite{Sil11}.
Si intende lasciare la possibilità di avere dati mancanti per permettere un'integrazione che sia compatibile con i sistemi attualmente in uso, in modo da implementare il sistema come \emph{data warehouse}, oltre a renderlo tollerante rispetto a eventuali omissioni di informazioni.

\section{Area concettuale della gravidanza}

\begin{figure}
    \centering
    \includesvg{vectors/modello-er-originale-gravidanza.svg}
    \caption{Schema Entità Relazione. Ingrandimento dell'area gravidanza.}
    \label{pregnancyerdiagram}
\end{figure}

La Figura \ref{pregnancyerdiagram} mostra l'ingrandimento dello schema Entità-Relazione per quanto riguarda le entità Paziente, Gravidanza e Malattia.

\subsubsection{Paziente}

Si introduce un identificativo per la paziente che risulta essere effettivamente univoco, ovvero il codice fiscale, assumendo che tutte le pazienti trattate abbiano un codice fiscale assegnato.
Ciò semplifica la rappresentazione dell'identificativo personale, che nello schema logico verrà inserito in quasi tutte le tabelle, rispetto alla tripla composta da nome, cognome e data di nascita, e previene i possibili, seppur rarissimi, casi di corrispondenza di questi dati per persone diverse.

\subsubsection{Gravidanza}

A questa entità fanno riferimento alcuni attributi che vengono registrati durante la visita del primo trimestre, oltre all'esito della gravidanza e alla data del primo ingresso.
Decidere se assegnare tali attributi a \enquote{Gravidanza} oppure a \enquote{Visita I° trimestre} può essere talvolta arbitrario: preferiamo quindi assegnare alla prima quelle informazioni che si possono considerare come proprie della gravidanza in sé (come la presenza di feti gemellari o il tipo di PMA), mentre alla seconda le informazioni sulla paziente che i si riferiscono al giorno o al periodo della visita (come ad esempio peso o altezza).

Identifichiamo due chiavi candidate, entrambe parziali in quanto l'entità è debole verso Paziente:
\begin{itemize}
\item la data del primo ingresso, attributo derivato e corrispondente alla data della prima visita effettuata, se ce n'è stata una, altrimenti alla data del parto;
\item la parità registrata all'inizio della gravidanza.
\end{itemize}

L'attributo \enquote{Parità} è composto.
Nei sistemi attualmente in uso è scritto solitamente in un unico campo, usando quindi la rappresentazione testuale che è calcolata come concatenazione del numero di figli nati a termine, del numero di figli nati pretermine, del numero di aborti (comprendente sia aborti spontanei sia interruzioni volontarie di gravidanza) e del numero totale di figli nati vivi, in questo ordine.

Sempre sull'attributo \enquote{Parità} si pone il seguente vincolo.
\begin{quote}
Se una paziente ha più di una gravidanza, per ogni coppia di gravidanze successive i valori degli attributi (non derivati) sono non decrescenti e almeno uno di essi è strettamente crescente.
\end{quote}

\subsubsection{Malattia}

L'entità \enquote{Malattia} rappresenta le patologie che possono presentarsi in concomitanza con la gravidanza.
Attualmente vengono registrate in modo sistematico solo per le pazienti seguite nell'ambulatorio \enquote{gravidanze a rischio} (vedi Sezione \ref{pregnanciesatrisk}) ma è applicabile alla gravidanza di qualsiasi paziente.
È importante considerare che le malattie possono variare tra le diverse gravidanze della stessa paziente.
Le informazioni riguardo alle eventuali terapie seguite sono memorizzate nell'attributo della relazione che lega gravidanza e malattie.

\section{Area concettuale delle visite}

\begin{figure}
    \centering
    \includesvg{vectors/modello-er-originale-visite.svg}
    \caption{Schema Entità Relazione. Ingrandimento dell'area visite.}
    \label{visitserdiagram}
\end{figure}

La Figura \ref{visitserdiagram} mostra l'ingrandimento dello schema Entità-Relazione per quanto riguarda le entità relative alle visite e all'entità Esame.

\section{Area concettuale del parto}

\begin{figure}
    \centering
    \includesvg{vectors/modello-er-originale-parto.svg}
    \caption{Schema Entità Relazione. Ingrandimento dell'area parto.}
    \label{deliveryerdiagram}
\end{figure}

\section{Area concettuale del neonato}

\begin{figure}
    \centering
    \includesvg{vectors/modello-er-originale-neonato.svg}
    \caption{Schema Entità Relazione. Ingrandimento dell'area neonato.}
    \label{newbornerdiagram}
\end{figure}

La Figura \ref{newbornerdiagram} mostra l'ingrandimento dello schema Entità-Relazione per quanto riguarda le entità Neonato e quelle relative alle misurazioni del cardiotocografo.