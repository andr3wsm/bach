\chapter{Conclusione}

Analizziamo velocemente le attività svolte per la costruzione della base di dati secondo i principi punti previsti nella Sezione \ref{realization}.

L'analisi della situazione di fatto, degli strumenti in uso e delle richieste di aggiunte ulteriori (Sezione \ref{analysis}) ha permesso la formalizzazione dei requisiti di un sistema complesso per la gestione del percorso gravidanza e parto.
Ciò ci ha permesso di scegliere una base di dati relazionale come strumento adatto alla risoluzione del problema specifico che viene posto.

Nei vari stadi della progettazione, passando da un piano concettuale (Sezione \ref{conceptual}) a uno logico (Sezione \ref{logical}) fino alla realizzazione concreta attraverso \emph{script} in linguaggio SQL (Sezione \ref{physical}) e all'esemplificazione delle funzionalità della base di dati (Sezione \ref{functionality}), abbiamo mostrato il processo di costruzione di una base di dati completa e pronta all'impiego.
Queste fasi di progettazione si susseguono a cascata in modo naturale e sono guidate da regole e principi che facilitano il raggiungimento di un risultato coerente, anche se in determinati casi è necessario operare delle scelte che, seppure possano aggiungere complicazioni nella gestione delle ridondanze e dell'integrità dei dati, possono agevolare l'interrogazione e l'uso da parte di utenti anche inesperti.

Ogni fase del processo di realizzazione è corredata di schemi, che seguono i formalismi specifici delle singole fasi di progettazione affrontate, insieme a vincoli aggiuntivi e alla spiegazione delle eventuali scelte operate.
La documentazione è necessaria non solo per lo sviluppo iniziale del sistema ma anche per l'utilizzo quotidiano, ad esempio per la formulazione di interrogazioni, e per la manutenzione, nel caso in cui emergano criticità, errori o nuovi requisiti da integrare.

% si potrebbe aggiungere altro