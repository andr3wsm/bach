\chapter{Conclusione}

\section{Sintesi della progettazione della base di dati}

Analizziamo velocemente le attività svolte per la costruzione della base di dati secondo i principi punti previsti nella Sezione \ref{realization}.

L'analisi della situazione di fatto, degli strumenti in uso e delle richieste di aggiunte ulteriori (Sezione \ref{analysis}) ha permesso la formalizzazione dei requisiti di un sistema complesso per la gestione del percorso gravidanza e parto.
Ciò ci ha permesso di scegliere una base di dati relazionale come strumento adatto alla risoluzione del problema specifico che viene posto.

Nei vari stadi della progettazione, passando da un piano concettuale (Sezione \ref{conceptual}) a uno logico (Sezione \ref{logical}) fino alla realizzazione concreta attraverso \emph{script} in linguaggio SQL (Sezione \ref{physical}) e all'esemplificazione delle funzionalità della base di dati (Sezione \ref{functionality}), abbiamo mostrato il processo di costruzione di una base di dati completa e pronta all'impiego.
Queste fasi di progettazione si susseguono a cascata in modo naturale e sono guidate da regole e principi che facilitano il raggiungimento di un risultato coerente, anche se in determinati casi è necessario operare delle scelte che, seppure possano aggiungere complicazioni nella gestione delle ridondanze e dell'integrità dei dati, possono agevolare l'interrogazione e l'uso da parte di utenti anche inesperti.

Ogni fase del processo di realizzazione è corredata di schemi, che seguono i formalismi specifici delle singole fasi di progettazione affrontate, insieme a vincoli aggiuntivi e alla spiegazione delle eventuali scelte operate.
La documentazione è necessaria non solo per lo sviluppo iniziale del sistema ma anche per l'utilizzo quotidiano, ad esempio per la formulazione di interrogazioni, e per la manutenzione, nel caso in cui emergano criticità, errori o nuovi requisiti da integrare.

\section{Applicazioni del nuovo sistema}

L'analisi dei dati raccolti a partire da gravidanze e parti è un tema emergente e che attira interesse verso di sé, specialmente insieme alla creazione di modelli di intelligenza artificiale e altri metodi \emph{data-driven} volti a studiare e prevedere l'andamento dei parametri fisiologici e delle anomalie.

Ad esempio nel 2014, con la pubblicazione della prima base di dati \emph{open source} con rilevazioni del cardiotocografo, Chudáček e altri \cite{Chu14} prevedono che l'uso di questi dati sarà utile per il processamento e infine l'analisi automatica dei tracciati CTG.
Anche altri studi hanno evidenziato l'importanza dell'analisi dei tracciati CTG, come Cahill e altri \cite{Cah18}, e in particolare con l'ausilio dei recenti sviluppi nel campo del \emph{deep learning}, come McCoy e altri \cite{McC25}.

Ulteriori ricerche sono state condotte grazie all'applicazione del \emph{deep learning} per ottenere modelli predittivi a partire dai dati raccolti durante la gravidanza, come ad esempio lo studio di Cersonsky e altri \cite{Cer23} sulla prevedibilità della mortalità al parto che annovera, tra le variabili analizzate, molte delle informazioni contenute nella nostra base di dati.

\section{Possibili sviluppi futuri}

L'obiettivo finale della realizzazione della base di dati è raggiungere una piena integrazione del nuovo sistema con l'ambiente preesistente in cui lavora il personale medico.
Il sistema così ottenuto potrebbe beneficiare dell'uso di un'insieme di strumenti per facilitare questa integrazione.
\begin{itemize}
\item Lo sviluppo di un'interfaccia grafica è fondamentale per l'uso della base di dati perché si possono implementare le interrogazioni in un \emph{back-end} programmato a priori e si rende possibile l'uso anche all'utente finale senza che gli sia richiesta la conoscenza del linguaggio SQL né della struttura tabellare di basso livello.
\item Anche per il programmatore può essere utile operare a livello più astratto rispetto alle tabelle descritte nell'implementazione (Sezione \ref{physical}). Si possono quindi definire un'API per facilitare inserimenti singoli o in blocco oppure delle viste per sintetizzare interrogazioni su dati aggregati. 
\item Per agevolare il popolamento della base di dati a partire dai sistemi attualmente in uso si possono definire \emph{script} o semplici programmi per la conversione dai questi formati a inserimenti nella nuova base di dati relazionale.
\end{itemize}